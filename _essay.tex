\documentclass[11pt]{article}

\setlength{\parindent}{0cm}
\usepackage{natbib}

\title{Review essay}
\author{Steven Abreu}

\begin{document}
\maketitle
\tableofcontents

\clearpage
\section{Introduction}

Goal: survey abstractions, models, formalisms in unconventional (neuromorphic) computing.


\clearpage
\section{Digital computing}

\subsection{Origins of computing}

\subsection{Models of computing}

\subsection{Computational complexity}

\subsection{Computing machinery}

\subsection{Programming and HCI}

\subsubsection{Programming Languages}

\subsection{Artificial intelligence}

\subsection{What is missing?}

\clearpage
\section{Quantum computing}

\clearpage
\section{Non-digital computing}

\clearpage
\section{Neuromorphic computing}

\subsection{Computational neuroscience}

\subsection{Models of neural dynamics}

\subsubsection{Artificial neural networks}

\subsubsection{Spiking neural networks}

\subsection{Models of computation}

\subsection{Hardware implementations}

\subsubsection{Sub-threshold CMOS}

\clearpage
\section{Misc (todo)}

Analog computing (Analogia 

*“Digital computers deal with integers, binary sequences, deterministic logic, and time that is idealized into discrete increments. Analog computers deal with real numbers, nondeterministic logic, and continuous functions, including time as it exists as a continuum in the real world.”*

“In analog computing, complexity resides in architecture, not code. Information is processed as continuous functions of values such as voltage and relative pulse frequency rather than by logical operations on discrete sequences of bits. To an analog computer, one bit here or there makes no great difference. To a digital computer, one bit can make all the difference in the world”

“Digital computing, intolerant of error or ambiguity, depends upon precise definitions and error correction at every step. Analog computing not only tolerates errors and ambiguities but learns to thrive on them. Digital computers, in a technical sense, are analog computers so hardened against noise that they have lost their immunity to it. Analog computers embrace noise: a real-world neural network, such as the visual or auditory system in a developing brain, requiring a certain level of background noise in order to work.”

“Analog computing is alive and well despite vacuum tubes being commercially extinct. It is advancing on two fronts: from the bottom up, driven by drone warfare, autonomous vehicles, and mobile devices pushing the development of neuromorphic, or neuron-like, microprocessors; and from the top down as networks formed by interconnected populations of digital computers turn to analog computation in their infiltration and control of the world”

“Say, for example, you build a system to map highway traffic in real time, by giving cars access to the map in exchange for reporting their own location and speed. The result is a decentralized control system. Nowhere is there any central controlling model besides the real-time reports, a real-time map, and a simple algorithm that chooses the shortest path in time between two points. The complexity is not in the algorithm; it’s in the traffic itself”

“A model of the social graph becomes the social graph, and proliferates across the world.”

\clearpage
\bibliographystyle{plain}
\bibliography{_JabRef.bib}

\end{document}